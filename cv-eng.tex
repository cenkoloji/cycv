% LaTeX Curriculum Vitae Template {{{
%
% Copyright (C) 2004-2009 Jason Blevins <jrblevin@sdf.lonestar.org>
% http://jblevins.org/projects/cv-template/
%
% You may use use this document as a template to create your own CV
% and you may redistribute the source code freely. No attribution is
% required in any resulting documents. I do ask that you please leave
% this notice and the above URL in the source code if you choose to
% redistribute this file.}}}

\documentclass[letterpaper]{article} % {{{

\usepackage{hyperref}
\usepackage{geometry}
\usepackage{verbatim}
\usepackage[utf8]{inputenc}
\usepackage[turkish]{babel}

% Comment the following lines to use the default Computer Modern font
% instead of the Palatino font provided by the mathpazo package.
% Remove the 'osf' bit if you don't like the old style figures.
\usepackage[T1]{fontenc}
\usepackage[sc,osf]{mathpazo}

% Set your name here
\def\name{Süleyman Cenk YILDIZ}

% Replace this with a link to your CV if you like, or set it empty
% (as in \def\footerlink{}) to remove the link in the footer:
\def\footerlink{http://jblevins.org/projects/cv-template/}

% The following metadata will show up in the PDF properties
\hypersetup{
  colorlinks = true,
  urlcolor = black,
  pdfauthor = {\name},
  pdfkeywords = {particle, physics, mathematics},
  pdftitle = {\name: Curriculum Vitae},
  pdfsubject = {Curriculum Vitae},
  pdfpagemode = UseNone
}

\geometry{
  body={6.5in, 8.5in},
  left=1.0in,
  top=1.25in
}

% Customize page headers
\pagestyle{myheadings}
\markright{\name}
\thispagestyle{empty}

% Custom section fonts
\usepackage{sectsty}
\sectionfont{\rmfamily\mdseries\Large}
\subsectionfont{\rmfamily\mdseries\itshape\large}

% Other possible font commands include:
% \ttfamily for teletype,
% \sffamily for sans serif,
% \bfseries for bold,
% \scshape for small caps,
% \normalsize, \large, \Large, \LARGE sizes.

% Don't indent paragraphs.
\setlength\parindent{0em}

% Make lists without bullets
\renewenvironment{itemize}{
  \begin{list}{}{
    \setlength{\leftmargin}{1.5em}
  }
}{
  \end{list}
}% }}}

\begin{document}

% Place name at left or center
{\huge \name} %\centerline{\huge \bf }
\vspace{0.25in}

%Contact Info {{{
\begin{comment} %{{{
\begin{minipage}{0.45\linewidth}
  \href{http://www.boun.edu.tr/}{Bogazici University} \\
  Department of Physics \\
  36360, Bebek, Istanbul
\end{minipage}
\begin{minipage}{0.45\linewidth}
  \begin{tabular}{ll}
    Mobile Phone: & (0090) 532 2668603  \\
    Office Phone: & (0090) 212 3597619  \\
    Fax:          &  (0090) 212 2872466 \\
    Email:        & \href{mailto:cenk.yildiz@boun.edu.tr}{\tt cenk.yildiz@boun.edu.tr} \\
    %Homepage: & \href{http://www.stat-or.unc.edu/}{\tt http://www.stat-or.unc.edu/} \\
  \end{tabular}
\end{minipage}\\
\end{comment}%}}}

\begin{minipage}{0.45\linewidth}
  \href{http://www.cern.ch/}{CERN} \\
  CH-1211 Genève 23,\\
  Switzerland
\end{minipage}
\begin{minipage}{0.45\linewidth}
  \begin{tabular}{ll}
    Mobile Phone: & (0041) 76 7583287\\
    Office Phone: & (0041) 22 7678934  \\
    Email:        & \href{mailto:cenk.yildiz@cern.ch}{\tt cenk.yildiz@cern.ch} \\
    %Homepage: & \href{http://www.stat-or.unc.edu/}{\tt http://www.stat-or.unc.edu/} \\
  \end{tabular}
\end{minipage}
%}}}

\section*{Introduction} % {{{

Young doctor of physics with 4 years experience in experimental particle physics, with
focus on the following areas: microMEGAs x-ray detectors(installation, hardware maintenance, data acquisition system, performance
tests, upgrades and analysis), vacuum systems, data acquisition and control systems.

During my doctoral studies, I had a lot of hands on experience, and have developed software/hardware solutions for problems
and tasks that are encountered in particle physics experiments. I'm proficient in computing, my skills being focused on
but not limited to C++ and python.

Being very interested in sharing knowledge, and possessing advanced communication skills, I was appointed as CAST
Experiment contact person.

I highly value collaborative spirit, and am skilled in working in collaborative environment efficiently.

%}}}

\begin{comment}
\section*{Education}%{{{

\begin{tabular}{ll}
2009 - 2013,& Ph.D. Physics, Bogazici University  \\
2005 - 2008,& M.S. Physics, Bogazici University   \\
2000 - 2005,& B.S. Physics, Koc University
\end{tabular}
%}}}
\end{comment}

\section*{Education and Qualifications} %{{{

\begin{tabular}{ll}

2009 - 2013 ,& \textsl{PhD. in Physics, Bogazici University \& CERN Axion Solar Telescope} \\
      & {\em Thesis Title:} Search for Axions with Micromegas Detectors in the CERN CAST Experiment\\
      & Supervisor: Prof. Metin Ar\i k\\\\

2005 - 2008,& \textsl{M.S. in Physics, Bogazici University}\\
      & {\em Thesis Title:} Unitary Matrix Hopf Algebras and Theta-Deformed Fermion Algebra\\
      & Supervisor: Prof. Metin Ar\i k \\\\

2000 - 2005 ,& \textsl{B.S. in Physics, Koc University} 

\begin{comment}  % {{{ Indep studies at Koç
\\\\
2005 Spring,& \textsl{Independent Study Project, Koc University}\\
      & Coherent States\\
      & Supervisor: Prof. Tekin Dereli\\\\

2004 Fall,& \textsl{Independent Study Project, Koc University} \\
      & Group Theory in Physics, Representations of the Rotation Group\\
      & Supervisor: Prof. Tekin Dereli\\\\

2004 Spring,& \textsl{Independent Study Project, Bogazici University} \\
      & The Quantum Mechanics of a Charged Particle in Uniform \\
      & Electric and Magnetic Fields and Landau Levels\\
      & Supervisor: Prof. Tekin Dereli\\\\
\end{comment}%}}}

\end{tabular} % }}}

\section*{Professional Experience}%{{{

\begin{tabular}{lp{0.8\textwidth}}

2009 - 2013 , & \textsl{CERN - CAST} Doctoral Student\\
             & As part of doctoral studies, I worked in the CAST experiment, spending most of
             my time in CERN. With main
             responsability being micromegas detectors, I have contributed to the hardware, software, and
             physics case of CAST. I took part in the organizational side, conducting run coordination, shift
             coordination duties, and I am contact person of CAST since 2011.\\\\

2009 - 2012, & \textsl{Bogazici University,} Teaching Assistant, Part-time\\
             &  Taught laboratory courses on introductory physics and electronics.\\\\
2003 - 2005 , & \textsl{Koc University,} Laboratory Research Assistant\\
              & Assisted laboratory experiments in the electronics laboratory.\\\\
2003 , & \textsl{Koc University,} Laboratory Assistant\\
             &  Assisted laboratory courses on introductory physics.\\\\
%2002 - 2003 , & \textsl{Koc University Volunteers, } Volunteer teaching assistant for economically disadvantaged high school students.\\
\end{tabular} %}}}

\begin{comment}
\section*{Teaching Experience}%{{{

\begin{tabular}{lp{0.8\textwidth}}
2009 - 2012, & \textsl{Bogazici University,} Teaching Assistant\\
2003 - 2005 , & \textsl{Koc University,} Laboratory Research Assistant\\
2003 , & \textsl{Koc University,} Laboratory Assistant for Introductory Physics courses\\
2002 - 2003 , & \textsl{Koc University Volunteers, } Volunteer teaching assistant for economically disadvantaged high school students.\\
\end{tabular} %}}}

\end{comment}

\section*{Selected Publications} % {{{

\begin{itemize}
    \item {\em Probing the eV-Mass Range for Solar Axions with CAST  }\\
    The CAST Collaboration, IEEE Nucl.Sci.Symp.Conf.Rec, 342-346, 2010.
    \item {\em Search for Sub-eV Mass Solar Axions by the CERN Axion Solar Telescope with 3He Buffer Gas} \\
    The CAST Collaboration, Phys. Rev. Lett., 107, 261302, 2011.
    \item {\em The New Micromegas X-ray Detectors in CAST}\\
    Tomás, A., {\em et al.}, X-Ray Spectrometry, 40, 240-246, 2011.
    \item {\em New Micromegas for Axion Searches in CAST}\\
    %Dafni, T., {\em et al.}, Nucl. Instrum. Meth. A, 628, 172-176, 2011.
    %\item {\em  Latest Results and Prospects of the CERN Axion Solar Telescope  } \\
    Irastorza, I., {\em et al.}, J. Phys.: Conf. Ser.,  309, 012001, 2011.
    %\item {\em Rare event searches based on Micromegas detectors: the T-REX project }\\
    %Dafni, T., {\em et al.}, J. Phys.: Conf. Ser.,  345, 022003, 2012.
    \item {\em Performance of Micromegas Detectors in the CAST Experiment} \\
    Yildiz, C., {\em et al.}, J. Phys.: Conf. Ser.,  347, 012029, 2012.
    \item {\em The Discrimination Capabilities of Micromegas Detectors at Low Energy}\\
    Iguaz, F.,{\em et al.}, Proceedings of TIPP2011, 37, 1079-1086, 2012.
    \item {\em  CAST Microbulk Micromegas in the Canfranc Underground Laboratory}\\
    Tomas, A., {\em et al.}, Proceedings of TIPP2011, 37, 478-482, 2012.
    \item {\em Future Axion Searches with the International Axion Observatory (IAXO)}\\
    Irastorza, I., {\em et al.}, J. Phys.: Conf. Ser.,  460, 012002, 2013.
    \item {\em Low-background X-ray Detection with Micromegas for Axion Research}\\
    Garcia, J.,  {\em et al.}, J. Phys.: Conf. Ser.,  460, 012003, 2013.
    %\item {\em CAST Solar Axion Search with $^3$He buffer gas: Closing the hot dark matter gap}\\
    %The CAST Collaboration, arXiv:1307.1985.
%    \item {\em  IAXO - The International Axion Observatory }\\
%    Vogel, J., {\em et al.}, 
\end{itemize}
%}}}

\section*{Talks/Posters} % {{{

\begin{itemize}
    \item {\em X-Ray Detectors of the CAST Experiment} \\
    13th Topical Seminar on Innovative Particle and Radiation Detectors, Siena, 2013

    \item {\em Performance of micromegas detectors in the CAST Experiment} \\
    2nd International Conference on Particle Physics, Istanbul, 2011
\end{itemize}
%}}}

\section*{Technical Skills} %{{{

\begin{itemize}
    \item Programming      : {\em C++, Python, ROOT, php, html, \LaTeX, Bash, Labview}
    \item Data Acquisition : {\em NIM, VME systems, National Instruments cards }
    \item Vacumm Systems   : {\em General vacuum knowledge, turbo Pumps, diaphram pumps}
    \item Operating Systems: {\em Linux, Windows, Android                       }
    \item Office Software  : {\em MS. Office, Libreoffice, Openoffice           }

\end{itemize}
%}}}
\begin{comment}
\end{comment}

\section*{Trainings and Courses} % {{{

\begin{tabular}{lp{0.8\textwidth}}
2013 June, & tCSC - Thematic CERN School of Computing on High Performance Computing, Split\\
2012 August, & Euroscipy - European Conference for Scientists Using Python - Advanced Tutorial, Brussels\\
2011 January , & ISTAPP - International School of Theory and Analysis in Particle Physics, Istanbul\\
2010 July , & Euroscipy - European Conference for Scientists Using Python - Basic Tutorial, Paris\\
2010 January , & ISOTDAQ - International School on Trigger and Data Acquisition, Ankara\\
2009 September, & LabVIEW Basics I-II Course, CERN, Geneva\\
\end{tabular}
%}}}

\section*{Language Skills} % {{{

\begin{itemize}
\item Turkish, \textsl{Native}
\item English, \textsl{Fluent}
\item French, \textsl{Intermediate}
\item Spanish, \textsl{Intermediate}
\end{itemize}
% }}}

\section*{Awards and Honors} % {{{

\begin{tabular}{ll}
2012,       & Berkol Doğan Award - Bogazici University Physics Department\\
2005-2007,  & TUBITAK Domestic Scholarship for Masters\\
2005,       & Top Ranking Student Award - Koc University Physics Department\\
2000-2005,  & Vehbi Koc Scholarship - Koc University
\end{tabular}
%}}}

\section*{Personal} % {{{

\begin{itemize}
\item Born on March 23, 1983.
\item Citizen of Republic of Turkey.
\end{itemize}

%}}}

\section*{Other}%{{{

\begin{tabular}{lp{0.8\textwidth}}
2011 - ... & \textsl{CAST Experiment, CERN,} Contactperson\\
%2009 - 2011 & \textsl{CAST Experiment, CERN,} Responsible of Sunset Micromegas Detectors and Slow Control Systems\\
2011 - ... & \textsl{CERN,} Official CERN Guide\\
\end{tabular} %}}}

\section*{References} %{{{

%References are available upon request

\begin{minipage}[l]{0.33\textwidth}
Serkant Çetin\\
Faculty of Arts and Sciences\\
Dogus University\\
34722 Acıbadem, Istanbul\\
scetin@dogus.edu.tr
\end{minipage}%
\hfill
\begin{minipage}[c]{0.33\textwidth}
Metin Arik\\
Department of Physics\\
Bogazici University\\
34342 Bebek, Istanbul\\
metin.arik@boun.edu.tr
\end{minipage}%
\hfill
\begin{minipage}[r]{0.33\textwidth}
Martyn Davenport\\
PH-DT Department\\
CERN 1211, Geneva 23\\
martyn.davenport@cern.ch\\
\end{minipage}%

% Put the references commented % }}}


\begin{comment}
\bigskip

% Footer
\begin{center}
\begin{footnotesize}
Last updated: \today \\
%\href{\footerlink}{\texttt{\footerlink}}
\end{footnotesize}
\end{center}
\end{comment}

\end{document}
