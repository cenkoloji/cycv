% LaTeX Curriculum Vitae Template {{{
%
% Copyright (C) 2004-2009 Jason Blevins <jrblevin@sdf.lonestar.org>
% http://jblevins.org/projects/cv-template/
%
% You may use use this document as a template to create your own CV
% and you may redistribute the source code freely. No attribution is
% required in any resulting documents. I do ask that you please leave
% this notice and the above URL in the source code if you choose to
% redistribute this file.}}}

\documentclass[letterpaper]{article} % {{{

\usepackage{longtable}

\usepackage{hyperref}
\usepackage{geometry}
\usepackage{verbatim}
\usepackage[utf8]{inputenc}
\usepackage[turkish]{babel}

% Comment the following lines to use the default Computer Modern font
% instead of the Palatino font provided by the mathpazo package.
% Remove the 'osf' bit if you don't like the old style figures.
\usepackage[T1]{fontenc}
\usepackage[sc,osf]{mathpazo}

% Set your name here
\def\name{Süleyman Cenk YILDIZ}

% Replace this with a link to your CV if you like, or set it empty
% (as in \def\footerlink{}) to remove the link in the footer:
\def\footerlink{http://jblevins.org/projects/cv-template/}

% The following metadata will show up in the PDF properties
\hypersetup{
  colorlinks = true,
  urlcolor = black,
  pdfauthor = {\name},
  pdfkeywords = {particle, physics, mathematics},
  pdftitle = {\name: Curriculum Vitae},
  pdfsubject = {Curriculum Vitae},
  pdfpagemode = UseNone
}

\geometry{
  body={6.5in, 8.5in},
  left=1.0in,
  top=1.0in
}

% Customize page headers
\pagestyle{myheadings}
\markright{\name}
\thispagestyle{empty}

% Custom section fonts
\usepackage{sectsty}
\sectionfont{\rmfamily\mdseries\Large}
\subsectionfont{\rmfamily\mdseries\itshape\large}

% Other possible font commands include:
% \ttfamily for teletype,
% \sffamily for sans serif,
% \bfseries for bold,
% \scshape for small caps,
% \normalsize, \large, \Large, \LARGE sizes.

% Don't indent paragraphs.
\setlength\parindent{0em}

% Make lists without bullets
\renewenvironment{itemize}{
  \begin{list}{}{
    \setlength{\leftmargin}{1.5em}
  }
}{
  \end{list}
}% }}}

\begin{document}

% Place name at left or center
{\huge \name} %\centerline{\huge \bf }
\vspace{0.25in}

%Contact Info {{{
\begin{comment} %{{{
\begin{minipage}{0.45\linewidth}
  \href{http://www.boun.edu.tr/}{Bogazici University} \\
  Department of Physics \\
  36360, Bebek, Istanbul
\end{minipage}
\begin{minipage}{0.45\linewidth}
  \begin{tabular}{ll}
    Mobile Phone: & (0090) 532 2668603  \\
    Office Phone: & (0090) 212 3597619  \\
    Fax:          &  (0090) 212 2872466 \\
    Email:        & \href{mailto:cenk.yildiz@boun.edu.tr}{\tt cenk.yildiz@boun.edu.tr} \\
    %Homepage: & \href{http://www.stat-or.unc.edu/}{\tt http://www.stat-or.unc.edu/} \\
  \end{tabular}
\end{minipage}\\
\end{comment}%}}}

\begin{minipage}{0.45\linewidth}
  Date of Birth: 23.03.1983\\
  Nationality: Turkish \\
\end{minipage}
\begin{minipage}{0.45\linewidth}
  \begin{tabular}{ll}
    Mobile Phone: & (0041) 76 7583287\\
    Office Phone: & (0041) 22 7674948  \\
    Email:        & \href{mailto:cenk.yildiz@cern.ch}{\tt cenk.yildiz@cern.ch} \\
    Home Address: & Chemin Rieu 6, 1208, Geneva, Switzerland\\
    Work Address: & CERN CH-1211 Geneve 23, Switzerland
    %Homepage: & \href{http://www.stat-or.unc.edu/}{\tt http://www.stat-or.unc.edu/} \\
  \end{tabular}
\end{minipage}
%}}}

\section*{Summary} % {{{

I am a Doctor of physics with 5 years experience in experimental particle physics, with focus on the following areas:
particle detectors(commissioning, installation, hardware maintenance, data acquisition, performance tests and analysis),
physics analysis, acquisition/monitoring/control systems and vacuum systems. During my experience at CERN, I have
succesfully finished all my projects/tasks and developed efficient solutions for problems that were encountered.\\
- I highly value collaborative spirit.\\
- I am skilled in working in collaborative environment in an efficient manner.\\
- I possess organizational and communication skills that large scale projects/experiments demand.

%}}}

\section*{Professional Experience}%{{{

\begin{longtable}{lp{0.82\textwidth}}

Feb, 2014 - ... , &  \textsl{CERN - Beam Line for Schools}, Project Associate\\
            &\href{http://home.web.cern.ch/students-educators/spotlight/2014/competition-beam-line-schools}{http://home.web.cern.ch/students-educators/spotlight/2013/competition-beam-line-schools} \\
             & Currently working at BL4S competition with following responsibilities:\\
             &- Detector Responsible: Main responsible of the detectors that will be used in the beam line, such as
             Delay Wire Chambers, Lead Glass Calorimeters, Scintillators, Cherenkov Counters. The responsibility
             consists of performance tests, calibration, installation and analysis of detectors. \\
             &- DAQ: Implemented a modular acquisition and monitoring software using NIM and VME systems on hardware,
             and ATLAS TDAQ Framework on the software side.\\
             &- Physics Analysis: Developed a C++ based analysis software using modern programming techniques.\\
             &- Optimization of the experiment: Did the feasibility study of winning experiment and suggested
             improvements to the setup for the best result.\\
             &- Organization: Organized meetings with experts, handled the gas installation, radioactive source loans
             and mechanical/electrical work.\\
             &- Documentation: Prepared detailed documentation about every aspect of the project to ensure the knowledge
             transfer for next years. \\\\

2009 - 2014 , & \textsl{CERN - CERN Axion Solar Telescope}, Doctoral Researcher\\
             & As part of my doctoral studies, I worked in the CAST experiment, spending most of my time in CERN.
             Responsibilities consisted of:\\
             &- Micromegas detectors: Took responsibility of 2 micromegas detectors for installation and maintenance, data analysis, data
             acquisition system, upgrades and vacuum system.\\
             &- CAST Slow Control System: Been the main responsible of the Labview based acquisition system for hardware
             maintenance, installation of new sensors and software upgrades.\\
             &- CFD Simulations: Run CFD simulations together with CERN EN-CV group, analysed and interpreted the results.\\
             &- Physics Analysis: Developed new methods for interpreting CAST data using results of the CFD simulations.\\
             &- Magnet movement system: Organized and analysed Grid/Survey measurements which assure the CAST magnets
             solar tracking accuracy, developed new analysis methods, assisted the Solar Filming.\\
             &- Shift/Run Coordination: Did the shift coordination during 2012, and run coordination during several periods of
             data taking as the on-call responsible of the experiment.\\
             &- Contact person: Trained official CERN guides at CAST, maintained the visitor area, updated CAST
             posters, maintained the CAST official website. \\\\

2009 - 2012, & \textsl{Bogazici University,} Teaching Assistant, Part-time\\
             &  Taught laboratory courses on introductory physics and electronics.\\\\
2003 - 2005 , & \textsl{Koc University,} Laboratory Research Assistant\\
              & Assisted laboratory experiments in the electronics laboratory.\\\\
2003 , & \textsl{Koc University,} Laboratory Assistant\\
             &  Assisted laboratory courses on introductory physics.\\\\
%2002 - 2003 , & \textsl{Koc University Volunteers, } Volunteer teaching assistant for economically disadvantaged high school students.\\
\end{longtable} %}}}

\section*{Education and Qualifications} %{{{

\begin{tabular}{ll}

2009 - 2013 ,& \textsl{PhD. in Physics, Bogazici University \& CERN Axion Solar Telescope} \\
      & {\em Thesis Title:} Search for Axions with Micromegas Detectors in the CERN CAST Experiment\\
      &\href{https://cds.cern.ch/record/1629526?ln=en}{[CERN-THESIS-2013-205]} \\
      & Supervisor: Prof. Metin Ar\i k\\\\

2005 - 2008,& \textsl{M.S. in Physics, Bogazici University}\\
      & {\em Thesis Title:} Unitary Matrix Hopf Algebras and Theta-Deformed Fermion Algebra\\
      & Supervisor: Prof. Metin Ar\i k \\\\

2000 - 2005 ,& \textsl{B.S. in Physics, Koc University} Graduated as top ranking student.

\begin{comment}  % {{{ Indep studies at Koç
\\\\
2005 Spring,& \textsl{Independent Study Project, Koc University}\\
      & Coherent States\\
      & Supervisor: Prof. Tekin Dereli\\\\

2004 Fall,& \textsl{Independent Study Project, Koc University} \\
      & Group Theory in Physics, Representations of the Rotation Group\\
      & Supervisor: Prof. Tekin Dereli\\\\

2004 Spring,& \textsl{Independent Study Project, Bogazici University} \\
      & The Quantum Mechanics of a Charged Particle in Uniform \\
      & Electric and Magnetic Fields and Landau Levels\\
      & Supervisor: Prof. Tekin Dereli\\\\
\end{comment}%}}}

\end{tabular} % }}}

\begin{comment}
\section*{Teaching Experience}%{{{

\begin{tabular}{lp{0.8\textwidth}}
2009 - 2012, & \textsl{Bogazici University,} Teaching Assistant\\
2003 - 2005 , & \textsl{Koc University,} Laboratory Research Assistant\\
2003 , & \textsl{Koc University,} Laboratory Assistant for Introductory Physics courses\\
2002 - 2003 , & \textsl{Koc University Volunteers, } Volunteer teaching assistant for economically disadvantaged high school students.\\
\end{tabular} %}}}
\end{comment}

\section*{Technical Skills} %{{{

\begin{itemize}
    \item Programming      : {\em C++, Python, ROOT, php, Html, \LaTeX, Bash}
    \item Data Acquisition : {\em NIM, VME, National Instruments hardware, Labview, ATLAS TDAQ Framework }
    \item Vacumm Systems   : {\em General vacuum knowledge, turbo Pumps, membrane pumps}
    \item Operating Systems: {\em Linux, Windows, Android                       }
    \item Office Software  : {\em MS. Office, Libreoffice, Openoffice           }

\end{itemize}
%}}}
\begin{comment}
\end{comment}

\section*{Trainings and Courses} % {{{

\begin{tabular}{lp{0.8\textwidth}}
2013 June, & tCSC - Thematic CERN School of Computing on High Performance Computing, Split\\
2012 August, & Euroscipy - European Conference for Scientists Using Python - Advanced Tutorial, Brussels\\
2011 January , & ISTAPP - International School of Theory and Analysis in Particle Physics, Istanbul\\
2010 July , & Euroscipy - European Conference for Scientists Using Python - Basic Tutorial, Paris\\
2010 January , & ISOTDAQ - International School on Trigger and Data Acquisition, Ankara\\
2009 September, & LabVIEW Basics I-II Course, CERN, Geneva\\
\end{tabular}
%}}}

\section*{Selected Publications} % {{{

\begin{itemize}
    \item {\em Probing the eV-Mass Range for Solar Axions with CAST  }\\
    The CAST Collaboration, IEEE Nucl.Sci.Symp.Conf.Rec, 342-346, 2010.
    \item {\em Search for Sub-eV Mass Solar Axions by the CERN Axion Solar Telescope with 3He Buffer Gas} \\
    The CAST Collaboration, Phys. Rev. Lett., 107, 261302, 2011.
    \item {\em The New Micromegas X-ray Detectors in CAST}\\
    Tomás, A., {\em et al.}, X-Ray Spectrometry, 40, 240-246, 2011.
    \item {\em New Micromegas for Axion Searches in CAST}\\
    %Dafni, T., {\em et al.}, Nucl. Instrum. Meth. A, 628, 172-176, 2011.
    %\item {\em  Latest Results and Prospects of the CERN Axion Solar Telescope  } \\
    Irastorza, I., {\em et al.}, J. Phys.: Conf. Ser.,  309, 012001, 2011.
    %\item {\em Rare event searches based on Micromegas detectors: the T-REX project }\\
    %Dafni, T., {\em et al.}, J. Phys.: Conf. Ser.,  345, 022003, 2012.
    \item {\em Performance of Micromegas Detectors in the CAST Experiment} \\
    Yildiz, C., {\em et al.}, J. Phys.: Conf. Ser.,  347, 012029, 2012.
    \item {\em The Discrimination Capabilities of Micromegas Detectors at Low Energy}\\
    Iguaz, F.,{\em et al.}, Proceedings of TIPP2011, 37, 1079-1086, 2012.
    \item {\em  CAST Microbulk Micromegas in the Canfranc Underground Laboratory}\\
    Tomas, A., {\em et al.}, Proceedings of TIPP2011, 37, 478-482, 2012.
    \item {\em Future Axion Searches with the International Axion Observatory (IAXO)}\\
    Irastorza, I., {\em et al.}, J. Phys.: Conf. Ser.,  460, 012002, 2013.
    \item {\em Low-background X-ray Detection with Micromegas for Axion Research}\\
    Garcia, J.,  {\em et al.}, J. Phys.: Conf. Ser.,  460, 012003, 2013.
    \item {\em CAST Solar Axion Search with $^3$He buffer gas: Closing the hot dark matter gap}\\
    The CAST Collaboration, Phys.Rev.Lett., 112, 091302, 2014.
%    \item {\em  IAXO - The International Axion Observatory }\\
%    Vogel, J., {\em et al.}, 
\end{itemize}
%}}}

\section*{Talks/Posters} % {{{

\begin{itemize}
    \item {\em X-Ray Detectors of the CAST Experiment} \\
    13th Topical Seminar on Innovative Particle and Radiation Detectors, Siena, 2013

    \item {\em Performance of micromegas detectors in the CAST Experiment} \\
    2nd International Conference on Particle Physics, Istanbul, 2011
\end{itemize}
%}}}

\section*{Language Skills} % {{{

\begin{itemize}
\item Turkish, \textsl{Native}
\item English, \textsl{Fluent}
\item French, \textsl{Intermediate}
\item Spanish, \textsl{Intermediate}
\end{itemize}
% }}}

\section*{Awards and Honors} % {{{

\begin{tabular}{ll}
2012,       & Berkol Doğan Award - Bogazici University Physics Department\\
2005-2007,  & TUBITAK Domestic Scholarship for Masters\\
2005,       & Top Ranking Student Award - Koc University Physics Department\\
2000-2005,  & Vehbi Koc Scholarship - Koc University
\end{tabular}
%}}}

\section*{Personal} % {{{

\begin{itemize}
\item A1 and B driving licence.
\item Official CERN guide since 2011.
\end{itemize}

%}}}

\begin{comment}
\section*{Other}%{{{

\begin{tabular}{lp{0.8\textwidth}}
2011 - ... & \textsl{CERN,} Official CERN Guide\\
\end{tabular} %}}}
\end{comment}

\section*{Interests} % {{{

\begin{itemize}
\item Rock climbing, Alpinism, Hiking, Skiing
\item Music, percussion instruments
\end{itemize}

%}}}

\section*{References} %{{{

References are available upon request

\begin{comment}
\begin{minipage}[l]{0.33\textwidth}
Serkant Çetin\\
Faculty of Arts and Sciences\\
Dogus University\\
34722 Acıbadem, Istanbul\\
scetin@dogus.edu.tr
\end{minipage}%
\hfill
\begin{minipage}[c]{0.33\textwidth}
Metin Arik\\
Department of Physics\\
Bogazici University\\
34342 Bebek, Istanbul\\
metin.arik@boun.edu.tr
\end{minipage}%
\hfill
\begin{minipage}[r]{0.33\textwidth}
Martyn Davenport\\
PH-DT Department\\
CERN 1211, Geneva 23\\
martyn.davenport@cern.ch\\
\end{minipage}%
\end{comment}

% Put the references commented % }}}

\begin{comment}
\bigskip

% Footer
\begin{center}
\begin{footnotesize}
Last updated: \today \\
%\href{\footerlink}{\texttt{\footerlink}}
\end{footnotesize}
\end{center}
\end{comment}

\end{document}
